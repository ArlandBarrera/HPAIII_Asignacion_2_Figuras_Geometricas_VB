En esta asignación se desarrolló un programa que calcula el área y el perímetro de dos figuras geométricas: el trapecio y el triángulo. El objetivo principal fue aplicar los conocimientos adquiridos en el curso de Herramientas de Programación Aplicada III, utilizando herramientas básicas de interfaz gráfica como Textbox para la entrada de datos y Label para mostrar los resultados. Este tipo de aplicación refuerza conceptos de geometría, así como habilidades de programación orientada a objetos y diseño de interfaces de usuario.

El programa fue diseñado para ser lo suficientemente flexible y fácil de usar, permitiendo que los usuarios ingresen las dimensiones correspondientes de cada figura geométrica y obtengan los resultados de manera inmediata. Además, se aplicaron fórmulas conocidas para el cálculo del área y el perímetro, utilizando tanto operaciones aritméticas simples como teoremas más complejos, como el de Pitágoras, en el caso del trapecio.