A través de la realización de este programa, se logró implementar con éxito las fórmulas matemáticas para el cálculo del área y el perímetro del trapecio y el triángulo. La interfaz gráfica del programa permitió una interacción sencilla con el usuario, mejorando la experiencia de uso mediante una clara disposición de los campos de entrada y la visualización de resultados.

Además, el proyecto reforzó la importancia de validar los datos de entrada. Por ejemplo, se implementaron controles para asegurar que la altura mayor del trapecio fuera siempre superior a la menor, evitando así resultados incorrectos. Este ejercicio permitió aplicar tanto principios de geometría como habilidades de programación esenciales para el desarrollo de software práctico.